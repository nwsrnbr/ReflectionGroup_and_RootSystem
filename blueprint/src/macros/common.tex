% In this file you should put all LaTeX macros and settings to be used both by
% the pdf version and the web version.
% This should be most of your macros.

% The theorem-like environments defined below are those that appear by default
% in the dependency graph. See the README of leanblueprint if you need help to
% customize this.
% The configuration below use the theorem counter for all those environments
% (this is what the [theorem] arguments mean) and never resets it.
% If you want for instance to number them within chapters then you can add
% [chapter] at the end of the next line.

\theoremstyle{definition}
\newtheorem{defi}{定義}[section]
\newtheorem{thm}[defi]{定理}
\newtheorem{prop}[defi]{命題}
\newtheorem{lem}[defi]{補題}
\newtheorem{cor}[defi]{系}
\newtheorem{rem}[defi]{注意}
\newtheorem{exa}[defi]{例}
\newtheorem{exe}[defi]{演習}
\newtheorem{fact}[defi]{事実}
\newtheorem*{defi*}{定義}
\newtheorem*{thm*}{定理}
\newtheorem*{prop*}{命題}
\newtheorem*{lem*}{補題}
\newtheorem*{cor*}{系}
\newtheorem*{rem*}{注意}
\newtheorem*{exa*}{例}
\newtheorem*{exe*}{演習}
\newtheorem*{fact*}{事実}
\newtheorem*{hosoku*}{補足}

\newcommand{\st}{\ \text{s.t.}\ }
\newcommand{\map}[2]{#1 \rightarrow #2}
\newcommand{\set}[2]{\left\lbrace #1 \mathrel{} \middle| \mathrel{} #2 \right\rbrace}
\newcommand{\ddv}[2]{\dfrac{d #1}{d #2}}
\newcommand{\dpdv}[2]{\dfrac{\partial #1}{\partial #2}}
\newcommand{\dsum}{\displaystyle \sum}
\newcommand{\dlim}{\displaystyle \lim}
\newcommand{\dint}{\displaystyle \int}
\newcommand{\abs}[1]{\left| #1 \right|}
\newcommand{\norm}[1]{\left\| #1 \right\|}
\newcommand{\inner}[2]{\left\langle #1, #2 \right\rangle}
\newcommand{\transpose}[1]{{}^t\!#1}
\newcommand{\C}{\mathbb{C}}
\newcommand{\R}{\mathbb{R}}
\newcommand{\Q}{\mathbb{Q}}
\newcommand{\Z}{\mathbb{Z}}
\newcommand{\N}{\mathbb{N}}
\newcommand{\V}{\mathbb{V}}
\newcommand{\I}{\mathbb{I}}
\newcommand{\F}{\mathbb{F}}
\newcommand{\M}{\operatorname{M}}
\newcommand{\GL}{\operatorname{GL}}
\newcommand{\SL}{\operatorname{SL}}
\newcommand{\tr}{\operatorname{tr}}
\newcommand{\Ker}{\operatorname{Ker}}
\newcommand{\Aut}{\operatorname{Aut}}
\newcommand{\End}{\operatorname{End}}
\newcommand{\ann}[1]{\operatorname{Ann}\, ({#1})}
\newcommand{\Co}{\operatorname{Co}}
\newcommand{\Fix}{\operatorname{Fix}}
\DeclareMathOperator{\id}{id}
\DeclareMathOperator*{\esssup}{ess\,sup}

% figureのキャプションをsectionに関連づける
\counterwithin{figure}{section}

% 式番号をsectionと関連づける
\numberwithin{equation}{section}

% 数式番号を参照されたもののみ表示
\mathtoolsset{showonlyrefs=true}

% 証明環境の見出しを日本語に
\def\proofname{\textbf{証明}}

% 証明環境の再定義
\makeatletter
\renewenvironment{proof}[1][\proofname]{\par
\pushQED{\qed}%
\normalfont \topsep6\p@\@plus6\p@\relax
\trivlist
\item\relax
{\bfseries
#1\@addpunct{\textbf{.}}}\hspace\labelsep\ignorespaces
}{
\popQED\endtrivlist\@endpefalse
}
\makeatother

% 証明環境の最後を黒四角に
\renewcommand{\qedsymbol}{$\blacksquare$}
